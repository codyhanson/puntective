\section{Related Work}

\cite{Stamatatos:2000:TGD:992730.992763} discusses a technique for classifying the `type' of publication that some text is from, depending on the word content and frequency. This approach was done with discriminant analysis, and a `training' dataset. It was not limited to humorous text.

In \cite{kaofunny}, a computational model for humor is presented, dealing specifically with homophone puns. The authors limited themselves to this subset in order to narrow the problem scope. This work shows that it is possible to 'quantify' something like word play, and should serve as a great knowledge base for my own research. These authors used the `Bag of Words' approach when doing their analysis.

\cite{ritchie2005computational} presents a method for pun generation using computers, as well as examines the `essence' and structure of different kinds of puns.

A chat bot for Yahoo Messenger is presented in \cite{humoristbot}, and one of the components is the ability to detect whether the human the bot is chatting with has said something humorous or not, so that the avatar can change and respond appropriately. The techniques used to identify humor in this case were looking for alliteration, antinomy (a contradiction between two beliefs or conclusions that are in themselves reasonable; a paradox.), and `adult slang' (presence of sexual words). These techniques were originally presented in \cite{COIN:COIN278}. In both of these works, input was limited to `one liners', short sentences that contain the entire joke. They also have a `learning' element to their algorithm, which needs a training dataset, and uses some Bayesian probability techniques.


