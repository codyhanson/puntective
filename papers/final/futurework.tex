\section{Future Work}
As can be seen by the inconclusive results of the simple heuristics that were implemented, additional and more complex heuristics will be required to be implemented to truly reveal the nature of humor in text, as recognized by humans. In order to be successful in this field, I think that computer science researches need to team up with those in the linguistics field, or be suitably well disciplined in both fields. Because words cannot often be looked at in isolation, advanced natural language processing techniques and algorithms that take into account the context of a word will be important to reliably classify phrases. Other heuristics that could be interesting to implement include more in depth definition comparisons, and `figure of speech' analysis.
Another interesting technique could be to automatically transcribe vocalized words into text, and do real time analysis on the fly. This may or may not be feasible depending on processing latency and number of heuristics being applied.

The acquisition of a larger dataset of puns and non-puns will also be required for future work. If possible an enriched dataset that includes metadata, such as the `setting' where the phrase was spoken, or other cultural cues which are important to the human understanding of humor.