
\section{Methodology and Heuristics}
They analysis process must first populate the pun graph with the phrase under test. The phrase is read in, and split on whitespace, with punctuation marks such as commas and periods being stripped. Then `word' nodes are created with `NEXT' edges between them. This is the skeleton, and core of the graph. 

Next a series of heuristics are applied, each having the pun graph at their disposal. In this implementation, the order of execution of heuristics is important, as the ones that come later have more nodes and edges to consider, and we can build up complexity as we go. It makes sense to execute more elemental analyses first (such as labeling parts of speech), since they don't  depend on anything from other heuristics.

The following sections describe some ideas for heuristics, some of which were implemented and tested in the results section.

\subsection{Odd Hyphenations}
A characteristic of puns that the authors have noticed is that they can sometimes contain `odd' hyphenations, or more precisely, hyphenations of words that are not commonly seen.

\begin{figure}[h]
\begin{mdframed}
  \emph{Atheism is a non-prophet organization.}
  \emph{A bicycle can't stand on its own because it is two-tired.}
  \caption{Puns with Odd Hyphenations}
 \label{oddhyphen}
\end{mdframed}
\end{figure}

In Figure~\ref{oddhyphen}, `non-prophet' (rather than `non-profit') and `two-tired' (rather than `too-tired') are hyphenations which most humans would deem as atypical. My idea would be to look for hyphenated phrases in a dictionary, or other datastore of figures of speech, and if the phrase isn't found, then we can perhaps make the assumption that it is a form of wordplay. 

\subsection{Homophones}
Puns often of homophones in them. A homophone is defined as being each of two or more words having the same pronunciation but different meanings, origins, or spelling, e.g., new and knew

\begin{figure}[h]
\begin{mdframed}
  \emph{What did the grape say when it got stepped on? Nothing - but it let out a little whine.}
  \caption{Pun with Homophones}
 \label{punhomophone}
\end{mdframed}
\end{figure}

In Figure~\ref{punhomophone}, the wordplay using homophones is `whine' which is a play on `wine' which is related to 'grape'. By looking for homophones of `whine', we would be led to `wine', which we could then make a linkage to grapes and winemaking, using a dictionary or encyclopedia.

\subsection{Dictionary Definitions}

\begin{figure}[h]
\begin{mdframed}
  \emph{I used to be addicted to soap, but I'm clean now.}
  \caption{Pun with terms linked by dictionary definitions}
 \label{addicted}
\end{mdframed}
\end{figure}

In Figure~\ref{addicted}, we can see some key words, `addicted', `clean' and `soap' have the connections between them that constitute a pun. For this example, we would process `addicted', and possibly find references to `clean', and the dictionary definition of `soap' could also have references to `clean'. When we see that `clean' has different definitions, with rather disparate meaning, we may be able to glean that there is a pun present.

\subsection{One letter mutations}

\begin{figure}[h]
\begin{mdframed}
  \emph{Did you see the movie about the hot dog? It was an Oscar Wiener.}
  \caption{Pun with relevant 1 letter mutation}
 \label{punmutation}
\end{mdframed}
\end{figure}

In Figure~\ref{punmutation} it can be seen that `Wiener'  is one letter removed from being the word `Winner', which is an appropriate reference to movies and the Oscar's awards show. By looking for other words which have a single letter different, it can help us to make linkages to the other parts of the sentence, using our other techniques. This could be efficiently computed using regular expressions, and a simple list of english words.

\subsection{Synthesized speech with dictionary search}
\begin{figure}[h]
\begin{mdframed}
  \emph{The roundest knight at king Arthur's round table was Sir Cumference.}
  \caption{Pun with audible word play}
 \label{sircumference}
\end{mdframed}
\end{figure}

In Figure~\ref{sircumference}, the word play of the phrase comes out when it is spoken. In this case, the Knight's name `Sir Cumference' becomes a homophone with `circumference'.
A stretch goal for this research could be to process text with text to speech software, and then back again from the generated audio, back into text. If the transformed text
differs from the input text, it could be indicative of audible word play. Of course this technique has its difficulties, and depends upon the quality of the text to speech, and speech to text synthesizers.

\section{A Reuseable Graph Data Structure}

An hypothesis that  we propose is that a graph data structure could be the best way for a computer program to catalogue and make use of connections between words in humorous text. As humans automatically maintain a mapping of incongruities, relationships, and connections between parts of a sentence, a graph could be constructed to represent the state of the analysis that a computer makes against a piece of text. 

The graph would have different types of nodes and edges, depending on what is being represented. For example, there would be a class of nodes that represent elements of the text under analysis, and there would also be nodes that represent `commonalities' between parts of the text. An example of a commonality could be a related subject between two words, such as `sports' between the elements `ball' and `score'. Another type of node could be 'parts of speech' information, which would allow different analytic modules to categorize parts of the sentence as verbs, adjectives, nouns, or as being part of a figure of speech. Adding this extra information could be required for more advanced humor detecting heuristics.

While in normal graphs, the edges don't necessarily carry much data of their own, the edges between nodes in this system are important because they define many different types of relationships between words. Some examples of different types of edges could be `homophone' edges denoting that two nodes are homophones of each other, or a 'figure of speech' edge which denotes that several nodes belong to another `figure of speech' node, denoting an idiomatic use of language. See Figure~\ref{pungraph} for an example of a Pun Graph that has been decorated by various heuristics.

\begin{center}
\begin{figure}[h]
  \includegraphics[keepaspectratio=true, scale=.35]{pun-graph-example.png}
  \caption{Concept - A Decorated Pun Graph}
   \label{pungraph}
\end{figure}
\end{center}

\subsection{Generalized scoring}
Because all edges and nodes inherit from the base classes that I will design, it makes the implementation of summing the contributions of all heuristics easy. To illustrate the general idea of how the system will use the graph, please refer to Algorithm~\ref{graphalgorithm}. The list of heuristics to be ran could potentially be a dynamically generated set, and the number of rounds of analysis could be bound by a timer or other upper limit, to ensure that the computation happens within a timely manner, or perhaps continues longer to become more rigorous.

\begin{algorithm}[h]
\begin{mdframed}
 \KwData{Input $text$}
 \KwData{Pun Graph $G$}
 \KwResult{Pun Score for the text, cumulatively assigned by all heuristics}
 Initialize graph $G$ with the Input $text$\;
 Initialize $score = 0$\;
 \While{there are heuristics left to run}{
      run heuristic on graph $G$\;
      update graph $G$ with additional nodes and edges\;
      next heuristic;
    }
  \For{$node \in G$}{
	 $score$ = $score$ + contribution of $node$\;
  }  
 \For{$edge \in G$}{
	 $score$ = $score$ + contribution of $edge$\;
  }  
  output $score$;
   \end{mdframed}
   \caption{Algorithm for integrating the Pun Graph with arbitrary heuristics}
 \label{graphalgorithm}
\end{algorithm}

\subsection{Motivation}
The idea of a graph to encode relationships between words and pieces of knowledge seems like an easy to grasp concept that has limitless potential for further research and extensibility. Any research could add their own types of data to a graph, and learn to take advantage of the encoded information in new and exciting ways.  

\section{Tools}
The language of choice for implementation is Python, due to its rich set of libraries, object oriented design support, and ability to create rapid prototypes. It is also well regarded in the scientific community, and easy to learn to use, without the intricacies of learning how to use a C or Java compiler. There is also an powerful library for Python called the Natural Language Toolkit \cite{NLTK}. This provides easy access to libraries that can be used for parsing sentences and building heuristics. It also provides an easy to use interface to Wordnet \cite{wordnet}.

For the graph data structure, rather than build my own graph implementation, I found that using an existing product called Neo4J to be advantageous. Neo4J \cite{neo4j} is a graph database that provides many benefits of a traditional SQL database, while being designed to encode data in nodes and edges, rather than tables. Neo4J provides a SQL-like query language called Cypher. This makes it easy to query for a subset of nodes and edges, and easily add to a graph. Neo4J is easily consumable via many programming languages, including Python.

\section{Datasets}
The input data for my development tasks is a small set of puns found on the internet, as well as a set of sentences that are not considered humorous. Due to the small size of the dataset, it is primarily intended for use as development test cases, rather than an exhaustive study or benchmark of this system.

The format of the data is JSON, so that it can be easily parsed by the analysis programs. JSON also makes it easy to add metadata to information, such as a list of tags that encode information about the phrase, as well as other properties that could be helpful. This is a more elegant technique than using a comma separated value file, and is less likely to degrade over time because of the use of key value pairs.
