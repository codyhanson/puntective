\section{Evaluation}

In order to judge the effectiveness of the system, we will compare the results of the pun classification to that of a human expert. We will build a dataset that contains a mixture of text snippets that contain puns, and some that do not. Also, because not all puns are created equal, we will seek out puns of varying complexity (as judged by the number of `connections' a human can make) to attempt to highlight limitations of our techniques. To ensure that the evaluation is fair, the texts in the dataset will be selected with as much randomness as possible, from a pool of candidates, and the algorithms will not be tailored in any way to a specific piece of text.

It will be important to track statistics on success rates, including false positives. False positives will indicate that the techniques are not refined enough, or are going down a path of incorrect analysis.

Speed of execution will not be a main goal or metric for this research. It is hopeful that search techniques will take at most on the order of a few minutes, but if the complexity of the analysis become of too high, those techniques will need to be rethought.

Another important piece of evaluation, although more subjective, will be to what degree the Pun Graph framework will be easily used and extended by other researchers.  The ideal use case will be for the researcher to extend one of the base or derived classes for edges and nodes to specifically have the properties that they wish to explore (perhaps different scoring, or a new type of relationship).

