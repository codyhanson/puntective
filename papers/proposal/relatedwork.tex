\section{Related Work}

\cite{Stamatatos:2000:TGD:992730.992763} discusses a technique for classifying the 'type' of publication that some text is from, depending on the word content and frequency. This approach was done with discriminant analysis, and a 'training' dataset. It was not limited to humorous text.

In \cite{kaofunny}, a computational model for humor is presented, dealing specifically with homophone puns. The authors limited themselves to this subset in order to narrow the problem scope. This work shows that it is possible to 'quantify' something like word play, and should serve as a great knowledge base for my own research. These authors used the 'Bag of Words' approach when doing their analysis.

\cite{ritchie2005computational} presents a method for pun generation using computers, as well as examines the 'essence' and structure of different kinds of puns.

