
\section{Proposed Methodology and Heuristics}
The first step will be to read in the words of the text under examination. The first naive approach will to just count the frequency of each word in a 'bag of words' approach. This will be simple to implement initially, and won't require deep knowledge of natural language processing (NLP) algorithms that would be required to parse the sentence into a tree-like structure. Once the bag of words is constructed we can apply several proposed heuristics.

If the word frequency counting proves to not be powerful enough to capture the essence of the text, NLP algorithms such as those demonstrated by \cite{stanfordparser} will be researched in order to augment the approach. 

In order to maximize development speed, and focus on actually solving the problems, we will use simple console based programs that leverage a scripting language such as Python. If needed, we will organize our backing data into a SQLite database, to minimize complexity and enable easy sharing with other academics. SQLite will also provide for a powerful querying mechanism while implementing code.

\subsection{Odd Hyphenations}
A characteristic of puns that the authors have noticed is that they can sometimes contain 'odd' hyphenations, or more precisely, hyphenations of words that are not commonly seen.

\begin{figure}[h]
  \caption{Puns with Odd Hyphenations}
  \emph{Atheism is a non-prophet organization.}
  \emph{A bicycle can't stand on its own because it is two-tired.}
 \label{oddhyphen}
\end{figure}

In Figure~\ref{oddhyphen}, 'non-prophet' (rather than 'non-profit') and 'two-tired' (rather than 'too-tired') are hyphenations which most humans would deem as atypical. My idea would be to look for hyphenated phrases in a dictionary, or other datastore of figures of speech, and if the phrase isn't found, then we can perhaps make the assumption that it is a form of wordplay. 

\subsection{Homophones}
Puns often of homophones in them. A homophone is defined as being each of two or more words having the same pronunciation but different meanings, origins, or spelling, e.g., new and knew

\begin{figure}[h]
  \caption{Pun with Homophones}
  \emph{What did the grape say when it got stepped on? Nothing - but it let out a little whine.}
 \label{punhomophone}
\end{figure}

In Figure~\ref{punhomophone}, the wordplay using homophones is 'whine' which is a play on 'wine' which is related to 'grape'. By looking for homophones of 'whine', we would be led to 'wine', which we could then make a linkage to grapes and winemaking, using a dictionary or encyclopedia.

\subsection{Dictionary Definitions}

\begin{figure}[h]
  \caption{Pun with terms linked by dictionary definitions}
  \emph{I used to be addicted to soap, but I'm clean now.}
 \label{addicted}
\end{figure}

In Figure~\ref{addicted}, we can see some key words, 'addicted', 'clean' and 'soap' have the connections between them that constitute a pun. For this example, we would process 'addicted', and possibly find references to 'clean', and the dictionary definition of 'soap' could also have references to 'clean'. When we see that 'clean' has different definitions, with rather disparate meaning, we may be able to glean that there is a pun present.

\subsection{One letter mutations}

\begin{figure}[h]
  \caption{Pun with relevant 1 letter mutation}
  \emph{Did you see the movie about the hot dog? It was an Oscar Wiener.}
 \label{punmutation}
\end{figure}

In Figure~\ref{punmutation} it can be seen that `Wiener'  is one letter removed from being the word 'Winner', which is an appropriate reference to movies and the Oscar's awards show. By looking for other words which have a single letter different, it can help us to make linkages to the other parts of the sentence, using our other techniques.

\subsection{Synthesized speech with dictionary search}
\begin{figure}[h]
  \caption{Pun with audible word play}
  \emph{The roundest knight at king Arthur's round table was Sir Cumference.}
 \label{sircumference}
\end{figure}

In Figure~\ref{sircumference}, the word play of the phrase comes out when it is spoken. In this case, the Knight's name 'Sir Cumference' becomes a homophone with 'circumference'.
A stretch goal for this research could be to process text with text to speech software, and then back again from the generated audio, back into text. If the transformed text
differs from the input text, it could be indicative of audible word play. Of course this technique has its difficulties, and depends upon the quality of the text to speech, and speech to text synthesizers.


