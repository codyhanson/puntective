\section{Introduction}

Written language is one of the crowning achievements of humans, and one of our most sophisticated and complex tools. An important part of this medium, is that of humor, and jokes. Humorous writing works to bring special joy to humans, and we are able to craft nuanced and smart jabs that require much thought to appreciate. What would it take to teach a computer the concept of 'humor'? How about just identifying its presence? An especially complex form of humor in spoken word and writing is the 'Pun'.  A pun, often referred to as a 'play on words' , and while easy for most humans to recognize, puns present a challenge to computers and the field of natural language processing,

In this paper we propose a system of heuristics and processing that attempts to recognize whether a piece of text constitutes a pun. By using techniques of natural language processing and heuristics for finding common properties of puns we will attempt to assign a 'confidence score' which will attempt to quantify the likelihood that the piece of text is a pun.

\section{Background}

\subsection{What makes a pun?}

By understanding common elements of puns, we can better train a computer to recognize them.