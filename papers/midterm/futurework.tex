\section{Future Work}

An interesting and valuable addition could be the rendering of the Pun Graph into an interactive image, where nodes and edges could be visualized, explored, and modified. The Graph could be output to a JSON format, which could be suitable for consumption by a graphing library such as D3 \cite{D3}.

Another interesting technique could be to automatically transcribe vocalized words into text, and do real time analysis on the fly. This may or may not be feasible depending on processing latency and number of heuristics being applied.

Finally, a `training database' of previously constructed pun graphs and their resulting scores could be useful in an advanced learning algorithm. By storing many samples of already processed humor, including confidence scores and relationships encoded in the Pun Graph, a learning agent could draw upon this data to more accurately predict if a new piece of text is humorous, and which other jokes and puns it is similar to.
